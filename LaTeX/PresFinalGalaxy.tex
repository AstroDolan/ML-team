\documentclass{beamer}
\usepackage{hyperref}
\usepackage[T1]{fontenc}
\usepackage[spanish,provide=*]{babel}
% Cargar apacite después de babel para citas autor-año correctas
\usepackage[notocbib]{apacite}

% other packages
\usepackage{latexsym,amsmath,xcolor,multicol,booktabs,calligra, tcolorbox}
\usepackage{graphicx,listings,stackengine}
% Manejo robusto de nombres de archivo con espacios y rutas de imágenes
\usepackage{grffile}
\graphicspath{{pic/}}

\renewcommand{\APACrefbtitle}[2]{#1}       % Muestra solo el título 
\renewcommand{\APACrefYearMonthDay}[3]{#1} % Muestra solo el año (omite mes/día)

\author{Brian Gonzalez, Dylan Jara}
\title{GAX-Kepler: Clasificador de morfología galáctica mediante el uso de redes neuronales convolucionales}
\subtitle{}
\institute{
    Universidad de Santiago de Chile
}
\date{\today}
\usepackage{Ritsumeikan}

% defs
\def\cmd#1{\texttt{\color{red}\footnotesize $\backslash$#1}}
\def\env#1{\texttt{\color{blue}\footnotesize #1}}
\definecolor{deepblue}{rgb}{0,0,0.5}
\definecolor{ccnuMainColor}{RGB}{57,64,73}
\definecolor{deepgreen}{rgb}{0,0.5,0}
\definecolor{halfgray}{gray}{0.55}

\lstset{
    basicstyle=\ttfamily\small,
    keywordstyle=\bfseries\color{deepblue},
    emphstyle=\ttfamily\color{ccnuMainColor},    % Custom highlighting style
    stringstyle=\color{deepgreen},
    numbers=left,
    numberstyle=\small\color{halfgray},
    rulesepcolor=\color{red!20!green!20!blue!20},
    frame=shadowbox,
}

\renewcommand{\APACrefbtitle}[2]{#1} % Muestra solo el título
\renewcommand{\APACrefYearMonthDay}[3]{#1} % Muestra solo el año
\renewcommand{\APACrefnote}[1]{} % Elimina las notas

\begin{document}

\begin{frame}
    \titlepage
    \vspace*{-0.6cm}
    \begin{figure}
        \begin{center}
                % Ajuste de tamaño más visible y uso de \graphicspath
                \includegraphics[keepaspectratio, width=0.25\textwidth]{Usach S1.png}
        \end{center}
    \end{figure}
\end{frame}

% Restaurar la tabla de contenidos y corregir el problema de 'Referencias32'
\begin{frame}
    \tableofcontents[sectionstyle=show, subsectionstyle=show/shaded/hide]
\end{frame}

%//////////////////////////////////////////////////////%
%/////////////////////////////////////////////%\
\section{Estado del Arte}

\begin{frame}{Estado del Arte: Evolución de la Clasificación}
    La clasificación morfológica ha evolucionado desde la inspección visual hasta el Deep Learning.
    
    \begin{itemize}
        \item \textbf{Dieleman et al. (2015):} El cambio de paradigma. Ganadores del desafío Kaggle Galaxy Zoo, demostraron que las CNNs pueden predecir probabilidades de morfología mejor que los humanos, utilizando rotaciones para invarianza \cite{dieleman2015rotation}.
        
        \item \textbf{Arquitecturas Híbridas (Actualidad):} Modelos recientes combinan la extracción de características profundas con clasificadores robustos como SVM/SVR o Random Forest para mejorar la generalización en datasets ruidosos \cite{huertas2015morphology}.
    \end{itemize}
\end{frame}

%/////////////////////////////////////////////%\
\section{Fundamentos Teóricos: SVR}

\begin{frame}{Support Vector Regression (SVR)}
        A diferencia de la clasificación tradicional, la SVR busca predecir valores continuos (probabilidades).
        
        \begin{itemize}
            \item \textbf{Objetivo:} Encontrar una función $f(x)$ que tenga, a lo sumo, una desviación $\epsilon$ de los objetivos reales $y_i$.
            \item \textbf{Margen de Tolerancia:} Ignora errores menores a $\epsilon$ (zona insensible), lo que la hace muy robusta ante el ruido \cite{drucker1996support}.
            \item \textbf{Kernel Trick:} Permite mapear datos a dimensiones superiores para resolver problemas no lineales \cite{vapnik1995nature}.
        \end{itemize}
\end{frame}

 
%/////////////////////////////////////////////%
\section{Implementación} %Buraian
\subsection{Arquitectura de la red}
\begin{frame}{Arquitectura del Modelo CNN}
\footnotesize
\begin{table}
\centering
\begin{tabular}{l c r}
\toprule
\textbf{Capa (tipo)} & \textbf{Forma de salida} & \textbf{Parámetros} \\
\midrule
Input (Sequential) & $(64, 64, 3)$ & 0 \\

Conv2D (64 filtros) & $(62, 62, 64)$ & 1,792 \\
Batch Normalization & $(62, 62, 64)$ & 256 \\
MaxPooling2D & $(31, 31, 64)$ & 0 \\

Conv2D (128 filtros) & $(29, 29, 128)$ & 73,856 \\
Batch Normalization & $(29, 29, 128)$ & 512 \\
MaxPooling2D & $(14, 14, 128)$ & 0 \\

Conv2D (256 filtros) & $(12, 12, 256)$ & 295,168 \\
Batch Normalization & $(12, 12, 256)$ & 1,024 \\
MaxPooling2D & $(6, 6, 256)$ & 0 \\
\midrule

\bottomrule
\end{tabular}
\end{table}
\end{frame}


\begin{frame}{Arquitectura del Modelo CNN II}
\footnotesize
\begin{table}
\centering
\begin{tabular}{l c r}
\toprule
\textbf{Capa (tipo)} & \textbf{Forma de salida} & \textbf{Parámetros} \\
\midrule

GlobalAveragePooling2D & (None, 256) & 0 \\
Dense & (None, 256) & 65,792 \\
Dropout & (None, 256) & 0 \\
Dense (Output) & (None, 37) & 9,509 \\
\midrule
\textbf{Total params} &  & \textbf{447,909} \\
Trainable params &  & 447,013 \\
Non-trainable params &  & 896 \\
\bottomrule
\end{tabular}
\end{table}
\end{frame}


%/////////////////////////////////////////////%
\section{Entrenamiento del modelo de estudio}
\begin{frame}[plain]{}
    \makebox[\textwidth][c]{%
        \includegraphics[width=1\textwidth]{pic/ArbolGalaxy.png}
    }
\end{frame}



%///////////////////////////////////////////////////%
\section{validación} %Brian%






%/////////////////////////////////////////////%
\section{Conclusiones} %Brian - Dylan
\begin{frame}{Conclusiones}
    \begin{enumerate}
        \item Las CNN son una herramienta poderosa para la clasificación de imágenes
        \item Con ayuda de un buen entendimiento en la teoría del problema, se pueden lograr muy buenos resultados.
        \item La clasificación automática de galaxias puede acelerar significativamente el análisis de grandes conjuntos de datos astronómicos. 
    \end{enumerate}
\end{frame}


%/////////////////////////////////////////////%
\begin{frame}[allowframebreaks]{Referencias}
    \tiny
    \bibliographystyle{apacite}
    \bibliography{PresFinalGalaxy}
\end{frame}

\end{document}