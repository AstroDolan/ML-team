\documentclass{beamer}
\usepackage{hyperref}
\usepackage[T1]{fontenc}
\usepackage[spanish]{babel}

% other packages
\usepackage{latexsym,amsmath,xcolor,multicol,booktabs,calligra, tcolorbox}
\usepackage{graphicx,listings,stackengine}

% dummy text; remove it when working on this template
\usepackage{lipsum}

\author{Brian Gonzalez, Dylan Jara}
\title{GAX-Kepler: Identificador de morfologia de galaxias mediante el uso de redes neuronales convolucionales}
\subtitle{}
\institute{
    Universidad de Santiago de Chile
}
\date{\today}
\usepackage{Ritsumeikan}

% defs
\def\cmd#1{\texttt{\color{red}\footnotesize $\backslash$#1}}
\def\env#1{\texttt{\color{blue}\footnotesize #1}}
\definecolor{deepblue}{rgb}{0,0,0.5}
\definecolor{ccnuMainColor}{RGB}{57,64,73}
\definecolor{deepgreen}{rgb}{0,0.5,0}
\definecolor{halfgray}{gray}{0.55}

\lstset{
    basicstyle=\ttfamily\small,
    keywordstyle=\bfseries\color{deepblue},
    emphstyle=\ttfamily\color{ccnuMainColor},    % Custom highlighting style
    stringstyle=\color{deepgreen},
    numbers=left,
    numberstyle=\small\color{halfgray},
    rulesepcolor=\color{red!20!green!20!blue!20},
    frame=shadowbox,
}


\begin{document}

\begin{frame}
    \titlepage
    \vspace*{-0.6cm}
    \begin{figure}[htpb]
        \begin{center}
            \includegraphics[keepaspectratio, scale=0.1]{pic/Usach S1.png}
        \end{center}
    \end{figure}
\end{frame}

\begin{frame}    
\tableofcontents[sectionstyle=show,
subsectionstyle=show/shaded/hide,
subsubsectionstyle=show/shaded/hide]
\end{frame}



%//////////////////////////////////////////////////////%
\section{Introducción} %Dylan%

\subsection{Introducción al problema}
\begin{frame}{Dataset Galaxy Zoo}
    \begin{itemize}
        \item<1-> En este trabajo, usamos Galaxy Zoo (Lintott et al., 2008, 2011; Willett et al., 2013) para la clasificacion y validacion de nuestros modelos de clasificacion.
        \item<2-> El projecto original Galaxy Zoo fue lanzado en julio de 2007 y obtuvo mas de 100,000 voluntarios en sus primeros diez dias.
        \item<3-> Los voluntarios clasificaron imagenes del Sloan Digital Sky Survey (SDSS) de galaxias en seis categorias: eliptica, espiral en sentido horario, espiral en sentido antihorario, edge-on , estrella/no lo se, o fusion de galaxias.
        \item<4-> 
    \end{itemize}
\end{frame}

\subsection{Justificacion}
\begin{frame}{¿De qué nos sirve?}
    \begin{itemize}
        \item This growing catalog of exoplanets has challenged the uniqueness of our solar system and informed theories of planet formation (Johnson, 2009).
        \item Basicamente facilita la exploracion espacial. Y esto ayuda a entender mejor el universo y el origen de la vida
    \end{itemize}
\end{frame}


%/////////////////////////////////////////////////////%
\section{Estado del arte} %Dylan%

\subsection{Tipos de galaxias} %Brian%
\begin{frame}{Definiciones}
    \begin{block}{Dominio}
        Debe ir categorizacion de galaxias con galaxy zoo
    \end{block}
    \begin{block}{Esquema de Relación}
        Deteccion usando IA
    \end{block}
    \begin{block}{Relación}
        Clasificacion multicategorica pero con binary-ce usando redes neuronales
    \end{block}
\end{frame}

\begin{frame}{Ejemplo de una imagen}
    \begin{figure}[H]
        \centering
        \includegraphics[width=6cm, height=6cm]{pic/google-relacional.jpg}
        \caption{Tablas de Clientes y Pedidos (Google Cloud)}
        \label{fig:enter-label}
    \end{figure}
\end{frame}


%///////////////////////////////////////////////////%
\section{Hipótesis} %Brian%
\subsection{¿Qué queremos lograr?}
aaa


%/////////////////////////////////////////////%
\section{Metodología} %Brian
\subsection{Arquitectura de la red}
aaa

\section{Ideas Futuras}
\begin{frame}
    \begin{itemize}
        \item Usar GPU para entrenar mas rapido
        \item Divide y venceras para distribuir
        \item Usar la CNN para trapasar a un Decision Tree o Random Forest
        \item Galaxy Zoo is a citizen science project which provides a visual morphological classification for nearly one million galaxies in its first phase (Galaxy Zoo 1) distinguishing elliptical from spiral galaxies.
        \item 
    \end{itemize}
\end{frame}

\section{Refs}
\begin{frame}
    \begin{itemize}
        \item We use scikit-learn (Pedregosa et al., 2011) python library to perform the experiments and procedures reported in this Section
        \item For a further analysis on the problem with two classes, we use the Receiver Operating Characteristic (ROC) curve and the Area Under the ROC
        curve (AUC, Bradley, 1997)
        \item Reducir la cantidad de clases existentes para disminuir la no-linealidad de los datos
        \item Throughout this work, we use Galaxy Zoo (Lintott et al., 2008, 2011; Willett et al., 2013) 
        classification as supervision and validation (ground truth) to our classification models.
        \item Several authors (Abraham et al., 1996; Conselice et al., 2000; Conselice, 2003; Lotz et al., 2004) studied and presented resultsabout 
        objective galaxy morphology measures with Concentration, Asymmetry, Smoothness, Gini, and M20 (CASGM system).
        \item The main purpose of this investigation is to answer the question “How to morphologically classify galaxies using Galaxy Zoo 
        (Lintott et al., 2008, 2011; Willett et al., 2013) classification through non-parametric features and Machine Learning methods?”
        \item Deep Convolutional Neural Network (CNN) is a well-established methodology to classify images (Goodfellow et al., 2016).
    \end{itemize}
\end{frame}

%/////////////////////////////////////////////%
\section{Conclusiones} %Brian - Dylan
\begin{frame}
    \begin{enumerate}
        \item Las bases de datos son esenciales en nuestra vida
        \item No son perfectas y necesitan manejo especializado
        \item Perdurarán hasta mucho tiempo más
    \end{enumerate}
\end{frame}

\end{document}