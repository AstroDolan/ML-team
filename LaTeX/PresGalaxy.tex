\documentclass{beamer}
\usepackage{hyperref}
\usepackage[T1]{fontenc}
\usepackage[spanish,provide=*]{babel}
% Cargar apacite después de babel para citas autor-año correctas
\usepackage{apacite}

% other packages
\usepackage{latexsym,amsmath,xcolor,multicol,booktabs,calligra, tcolorbox}
\usepackage{graphicx,listings,stackengine}
% Manejo robusto de nombres de archivo con espacios y rutas de imágenes
\usepackage{grffile}
\graphicspath{{pic/}}

\renewcommand{\APACrefbtitle}[2]{#1}       % Muestra solo el título 
\renewcommand{\APACrefYearMonthDay}[3]{#1} % Muestra solo el año (omite mes/día)

\author{Brian Gonzalez, Dylan Jara}
\title{GAX-Kepler: Clasificador de morfología galáctica mediante el uso de redes neuronales convolucionales}
\subtitle{}
\institute{
    Universidad de Santiago de Chile
}
\date{\today}
\usepackage{Ritsumeikan}

% defs
\def\cmd#1{\texttt{\color{red}\footnotesize $\backslash$#1}}
\def\env#1{\texttt{\color{blue}\footnotesize #1}}
\definecolor{deepblue}{rgb}{0,0,0.5}
\definecolor{ccnuMainColor}{RGB}{57,64,73}
\definecolor{deepgreen}{rgb}{0,0.5,0}
\definecolor{halfgray}{gray}{0.55}

\lstset{
    basicstyle=\ttfamily\small,
    keywordstyle=\bfseries\color{deepblue},
    emphstyle=\ttfamily\color{ccnuMainColor},    % Custom highlighting style
    stringstyle=\color{deepgreen},
    numbers=left,
    numberstyle=\small\color{halfgray},
    rulesepcolor=\color{red!20!green!20!blue!20},
    frame=shadowbox,
}

\renewcommand{\APACrefbtitle}[2]{#1} % Muestra solo el título
\renewcommand{\APACrefYearMonthDay}[3]{#1} % Muestra solo el año
\renewcommand{\APACrefnote}[1]{} % Elimina las notas

\begin{document}

\begin{frame}
    \titlepage
    \vspace*{-0.6cm}
    \begin{figure}
        \begin{center}
                % Ajuste de tamaño más visible y uso de \graphicspath
                \includegraphics[keepaspectratio, width=0.25\textwidth]{Usach S1.png}
        \end{center}
    \end{figure}
\end{frame}

% Restaurar la tabla de contenidos y corregir el problema de 'Referencias32'
\begin{frame}
    \tableofcontents[sectionstyle=show, subsectionstyle=show/shaded/hide]
\end{frame}

%//////////////////////////////////////////////////////%
\section{Introducción} %Dylan%
\subsection{Introducción al problema}
\begin{frame}{Dataset Galaxy Zoo}
    \begin{itemize}
        \item <1->Una galaxia es un entorno en el cual las estrellas nacen y mueren, y las galaxias lejanas son faros luminosos que nos permiten explorar el universo lejano. \cite{SILK2001337}
        \item<1-> En este trabajo, usamos Galaxy Zoo \cite{lintott2008galaxy,willett2013galaxy} para la clasificacion y validacion de nuestros modelos de clasificacion.
        \item<2-> El projecto original Galaxy Zoo fue lanzado en julio de 2007 y obtuvo mas de 100,000 voluntarios en sus primeros diez dias.
        \item<3-> Los voluntarios clasificaron imagenes del Sloan Digital Sky Survey (SDSS) de galaxias en seis categorias: eliptica, espiral en sentido horario, 
        espiral en sentido antihorario, edge-on , estrella/no lo se, o fusion de galaxias. 
    \end{itemize}
\end{frame}

\subsection{Justificacion}
\begin{frame}{¿De qué nos sirve?}
    \begin{itemize}
        \item La clasificacion morfologica de galaxias es fundamental para entender la formacion y evolucion de las galaxias.
        \item Basicamente facilita la exploracion espacial. Y esto ayuda a entender mejor el universo y el origen de la vida.
        \item 
    \end{itemize}
\end{frame}

\subsection{Estado del arte} %Dylan%
\begin{frame}{Estado del arte}
    \begin{itemize}
        \item<1->La clasificacion morfologica de galaxias es una pieza clave de informacion para definir muestras de galaxias con el objetivo de 
        estudiar la estructura a gran escala del universo. \cite{barchi2020machine}
        \item<2-> Muchos autores \cite{10.1093/mnras/279.3.L47, conselice2000asymmetry} han estudiado y presentado resultados sobre medidas objetivas de morfologia de 
        galaxias con Concentracion, Asimetria, Suavidad, Gini y M20 (sistema CASGM).
        \item<3-> Las Redes convolucionales profundas (CNN) son una metodología bien establecida para clasificar imágenes \cite{goodfellow2016deep}.
    \end{itemize}
\end{frame}

 
%/////////////////////////////////////////////////////%
\section{Tipos de galaxias} %Brian%
\begin{frame}{Tipos de galaxias}

\makebox[\textwidth][c]{%
\begin{minipage}{0.9\textwidth}
\centering

\begin{minipage}{0.28\textwidth}
    \centering
    \includegraphics[width=\linewidth,height=4cm,keepaspectratio]{pic/M61espiral.jpg}

    \vspace{0.2cm}
    \textbf{Galaxia Espiral}\\
    Galaxia espiral M61. Cúmulo de Virgo.
\end{minipage}
\hfill
\begin{minipage}{0.28\textwidth}
    \centering
    \includegraphics[width=\linewidth,height=4cm,keepaspectratio]{pic/M32eliptica.jpg}

    \vspace{0.2cm}
    \textbf{Galaxia Elíptica}\\
    Galaxia elíptica Messier 32. Constelación de Andrómeda.
\end{minipage}
\hfill
\begin{minipage}{0.28\textwidth}
    \centering
    \includegraphics[width=\linewidth,height=4cm,keepaspectratio]{pic/UGC4459.jpg}

    \vspace{0.2cm}
    \textbf{Galaxia Irregular}\\
    Galaxia enana UGC 4459. Constelación Osa Mayor.
\end{minipage}

\end{minipage}
}
\end{frame}


%/////////////////////////////////////////////%
\section{Descripción de los datos de estudio}
\begin{frame}[plain]{}
    \makebox[\textwidth][c]{%
        \includegraphics[width=1\textwidth]{pic/ArbolGalaxy.png}
    }
\end{frame}

\begin{frame}[plain]{Probabilidades de Q4}
    \makebox[\textwidth][c]{%
        \includegraphics[width=1\textwidth]{pic/outputQ4.png}
    }
\end{frame}

\begin{frame}[plain]{Probabilidades de Q11}
    \makebox[\textwidth][c]{%
        \includegraphics[width=1.1\textwidth]{pic/outputQ11.png}
    }
\end{frame}


%///////////////////////////////////////////////////%
\section{Hipótesis} %Brian%
\subsection{¿Qué queremos lograr?}
\begin{frame}{¿Qué queremos lograr?}
    \begin{itemize}
        \item Una red neuronal convolucional que pueda aprender representaciones jerárquicas de imágenes de galaxias y predecir de manera efectiva las probabilidades de sus características morfológicas con un bajo error cuadrático medio % menor a 0.007. >:D
    \end{itemize}
\end{frame}



%/////////////////////////////////////////////%
\section{Metodología} %Buraian
\subsection{Arquitectura de la red}
\begin{frame}{Arquitectura del Modelo CNN}
\footnotesize
\begin{table}
\centering
\begin{tabular}{l c r}
\toprule
\textbf{Capa (tipo)} & \textbf{Forma de salida} & \textbf{Parámetros} \\
\midrule
Input (Sequential) & $(64, 64, 3)$ & 0 \\

Conv2D (64 filtros) & $(62, 62, 64)$ & 1,792 \\
Batch Normalization & $(62, 62, 64)$ & 256 \\
MaxPooling2D & $(31, 31, 64)$ & 0 \\

Conv2D (128 filtros) & $(29, 29, 128)$ & 73,856 \\
Batch Normalization & $(29, 29, 128)$ & 512 \\
MaxPooling2D & $(14, 14, 128)$ & 0 \\

Conv2D (256 filtros) & $(12, 12, 256)$ & 295,168 \\
Batch Normalization & $(12, 12, 256)$ & 1,024 \\
MaxPooling2D & $(6, 6, 256)$ & 0 \\
\midrule

\bottomrule
\end{tabular}
\end{table}
\end{frame}


\begin{frame}{Arquitectura del Modelo CNN II}
\footnotesize
\begin{table}
\centering
\begin{tabular}{l c r}
\toprule
\textbf{Capa (tipo)} & \textbf{Forma de salida} & \textbf{Parámetros} \\
\midrule

GlobalAveragePooling2D & (None, 256) & 0 \\
Dense & (None, 256) & 65,792 \\
Dropout & (None, 256) & 0 \\
Dense (Output) & (None, 37) & 9,509 \\
\midrule
\textbf{Total params} &  & \textbf{447,909} \\
Trainable params &  & 447,013 \\
Non-trainable params &  & 896 \\
\bottomrule
\end{tabular}
\end{table}
\end{frame}

\begin{frame}[plain]{Esquema Metodológico}
    \makebox[\textwidth][c]{%
        \includegraphics[width=1.1\textwidth]{EsqMet.png}
    }
\end{frame}


%/////////////////////////////////////////////%
\section{Conclusiones e Ideas Futuras} %Brian - Dylan
\begin{frame}{Ideas Futuras}
    \begin{itemize}
        \item Usar GPU para entrenar más rápido.
        \item Método Divide y vencerás para distribuir.
        \item Usar la CNN para extración de datos y clasificar con Decision Tree, SVM o Random Forest.
    \end{itemize}
\end{frame}
\begin{frame}{Conclusiones}
    \begin{enumerate}
        \item Las CNN son una herramienta poderosa para la clasificación de imágenes
        \item Con ayuda de un buen entendimiento en la teoría del problema, se pueden lograr muy buenos resultados.
        \item La clasificación automática de galaxias puede acelerar significativamente el análisis de grandes conjuntos de datos astronómicos. 
    \end{enumerate}
\end{frame}


%/////////////////////////////////////////////%
\begin{frame}[allowframebreaks]{Referencias}
    \tiny
    \bibliographystyle{apacite}
    \bibliography{PresGalaxy}
    
\end{frame}

\end{document}