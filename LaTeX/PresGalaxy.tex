\documentclass{beamer}
\usepackage{hyperref}
\usepackage[T1]{fontenc}

% other packages
\usepackage{latexsym,amsmath,xcolor,multicol,booktabs,calligra, tcolorbox}
\usepackage{graphicx,listings,stackengine}

% dummy text; remove it when working on this template
\usepackage{lipsum}

\author{Brian Gonzalez, Dylan Jara}
\title{GAX-Kepler: Identificador de morfologia de galaxias mediante el uso de redes neuronales convolucionales}
\subtitle{}
\institute{
    Universidad de Santiago de Chile
}
\date{\today}
\usepackage{Ritsumeikan}

% defs
\def\cmd#1{\texttt{\color{red}\footnotesize $\backslash$#1}}
\def\env#1{\texttt{\color{blue}\footnotesize #1}}
\definecolor{deepblue}{rgb}{0,0,0.5}
\definecolor{ccnuMainColor}{RGB}{57,64,73}
\definecolor{deepgreen}{rgb}{0,0.5,0}
\definecolor{halfgray}{gray}{0.55}

\lstset{
    basicstyle=\ttfamily\small,
    keywordstyle=\bfseries\color{deepblue},
    emphstyle=\ttfamily\color{ccnuMainColor},    % Custom highlighting style
    stringstyle=\color{deepgreen},
    numbers=left,
    numberstyle=\small\color{halfgray},
    rulesepcolor=\color{red!20!green!20!blue!20},
    frame=shadowbox,
}


\begin{document}

\begin{frame}
    \titlepage
    \vspace*{-0.6cm}
    \begin{figure}[htpb]
        \begin{center}
            \includegraphics[keepaspectratio, scale=0.1]{pic/Usach S1.png}
        \end{center}
    \end{figure}
\end{frame}

\begin{frame}    
\tableofcontents[sectionstyle=show,
subsectionstyle=show/shaded/hide,
subsubsectionstyle=show/shaded/hide]
\end{frame}



%//////////////////////////////////////////////////////%
\section{Introducción}

\subsection{Planteamiento}
\begin{frame}{Introducción al problema}
    \begin{itemize}
        \item<1-> Usaremos una CNN con time series. Hay que aprender de eso
        \item<2-> Dataset de kepler, con curvas de luz
        \item<3-> Se debe explicar exoplanetas, curvas de luz. Las CNN se explican en metodologia
    \end{itemize}
\end{frame}

\subsection{Justificacion}
\begin{frame}{¿De qué nos sirve?}
    \begin{itemize}
        \item This growing catalog of exoplanets has challenged the uniqueness of our solar system and informed theories of planet formation (Johnson, 2009).
        \item Basicamente facilita la exploracion espacial. Y esto ayuda a entender mejor el universo y el origen de la vida
    \end{itemize}
\end{frame}


%/////////////////////////////////////////////////////%
\section{Estado del arte}

\subsection{Charles Williams Bachman}
\begin{frame}{Definiciones}
    \begin{block}{Dominio}
        Debe ir descubrimiento de exoplanetas con kepler y otras misiones
    \end{block}
    \begin{block}{Esquema de Relación}
        Deteccion usando IA
    \end{block}
    \begin{block}{Relación}
        Clasificacion usando redes neuronales
    \end{block}
\end{frame}

\begin{frame}{Ejemplo de una imagen}
    \begin{figure}[H]
        \centering
        \includegraphics[width=6cm, height=6cm]{pic/google-relacional.jpg}
        \caption{Tablas de Clientes y Pedidos (Google Cloud)}
        \label{fig:enter-label}
    \end{figure}
\end{frame}


%///////////////////////////////////////////////////%
\section{Hipótesis}
\subsection{¿Qué queremos lograr?}
aaa


%/////////////////////////////////////////////%
\section{Metodología}
\subsection{Arquitectura de la red}
aaa


%/////////////////////////////////////////////%
\section{Conclusiones}
\begin{frame}
    \begin{enumerate}
        \item Las bases de datos son esenciales en nuestra vida
        \item No son perfectas y necesitan manejo especializado
        \item Perdurarán hasta mucho tiempo más
    \end{enumerate}
\end{frame}

\end{document}